\documentclass[reqno]{article}
\usepackage{ae} % or {zefonts}
\usepackage[T1]{fontenc}
\usepackage[ansinew]{inputenc}
\usepackage{amsmath}
\usepackage{graphicx}
\usepackage{color}
\usepackage[colorlinks]{hyperref}
\setlength{\topmargin}{-1.4cm} \setlength{\oddsidemargin}{1cm} \setlength{\evensidemargin}{0.4cm}
\setlength{\textwidth}{15.6cm} \setlength{\textheight}{24.0cm}

\begin{document}

\begin{center}
  \begin{Large}
     RASP parser
   \end{Large}
\end{center}


RASP (Robust Accurate Statistical Parsing) is a robust parsing system for English. The GATE wrapper calls RASP as an 
external program, passing plain texts and shell script runall.sh as input. The Shell scripts runall.sh then invokes  
the tokeniser/tagger/morph/parser on raw texts. Output of the RASP are saved as "RASPToken" annotations and 
grammatical relation annotations. 
\\


\begin{itemize}
\item Features of the RASPToken:
        \begin{itemize}
           \item[i)] id: id of the token;
           \item[ii)] POS: the part-of-speech tag of the token;
           \item[iii)]Morph: the base word of the token;
           \item[iv)] content: the string image of the token;
           \item[v)]  length: the length of the token string.
           \\
        \end{itemize}


\item Grammatical relation annotations:
          \begin{itemize}
            \item[i)] type: grammatical relation type;
            \item[ii)] head: head token of the relationship;
            \item[iii)] dependant: dependant token of the relationship;
            \item[iv)] other parameter: other parameters of the relationship.
            \\
          \end{itemize}

\item Runtime parameters:
         \begin{itemize}
            \item[i)]document: the document to be processed
            \item[ii)]annotationSetName: the annotation set to be used for generated annotations
            \item[iii)]raspscript: a URL indicating the location of the RASP shell script
 runall.sh
            \\
          \end{itemize}


      
\item Platform dependability:
      RASP is only supported for Linux operating systems. Trying to run it on any other operating systems will generate an
      exception with the message: The RASP cannot be run on any other operating systems except Linux.            
      \\
      
\item Requirements:
      RASP is availabel from www.informatics.susx.ac.uk/research/nlp/carroll/rasp2-binary.tgz. It must be correctly 
      installed on the same machine as GATE. Before trying to run scripts for the first time, edit them to insert the 
      appropriate value for the shell variable RASP, which should be the file system pathname where you have installed 
      the RASP tools.     
      \\
      
\end{itemize}
      


\end{document}
